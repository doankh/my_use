\section{Object Types}
\subsection{Equality}
$=(y:\mathit{OclAny}):\mathit{Boolean}$ represents equality
between objects. It evaluates to true if $\mathit{self}$ is the same as $y$.\\
{\bf Notation:} $\mathit{self}$\verb+=+$y$
\subsection{Inequality}
$<>(y:\mathit{OclAny}):\mathit{Boolean}\stackrel{\mathit{def}}{=}\mathit{not}\ (
\mathit{self}=y)$ represents inequality between objects.\\
{\bf Notation:} $\mathit{self}$\verb+<>+$y$
\subsection{isUndefined}
$\mathit{isUndefined}():\mathit{Boolean}$ evaluates to true, if the
callee is undefined.\\
{\bf Notation:} $\mathit{self}$\verb+.isUndefined()+
\subsection{oclIsNew}
$\mathit{oclIsNew}():\mathit{Boolean}\stackrel{\mathit{def}}{=}
\mathit{self}@\mathit{pre}.\mathit{isUndefined}()$ can only be used
in a postcondition and states that the object has
been newly created during the execution of an operation.\\
{\bf Notation:} $\mathit{self}$\verb+.oclIsNew()+
\subsection{oclAsType}
$\mathit{oclAsType}(T):T$ is a ``cast'' or ``retyping'' expression,
  evaluating to the value of the callee, if it is an instance of $T$,
  and to $\mathit{Undefined}$ otherwise.\\
  {\bf Notation:} $\mathit{self}$\verb+.oclAsType(+$T$\verb+)+
\subsection{oclIsTypeOf}
$\mathit{oclIsTypeOf}(T):\mathit{Boolean}$ evaluates to $\mathit{true}$ if the
callee is an instance of type $T$.\\
{\bf Notation:} $\mathit{self}$\verb+.oclIsTypeOf(+$T$\verb+)+
\subsection{oclIsKindOf}
$\mathit{oclIsKindOf}(T):\mathit{Boolean}$ evaluates to $\mathit{true}$ if the
callee is
  an instance of type $T$ or one of $T$s subtypes, that is, the callee
  conforms to the type $T$.\\
  {\bf Notation:} $\mathit{self}$\verb+.isKindOf(+$T$\verb+)+
\section{Boolean Types}
\begin{tabular}{cc|ccccc}\hline
  $a$ & $b$ & $\mathit{not}\ b$ & $a\ \mathit{and}\ b$ & $a\
  \mathit{or}\ b$ & $a\ \mathit{implies}\ b$ & $a\ \mathit{xor}\ b$\\ \hline
  $\mathit{true}$ & $\mathit{true}$ & $\mathit{false}$ & $\mathit{true}$ & $\mathit{true}$ & $\mathit{true}$ & $\mathit{false}$ \\
  $\mathit{true}$ & $\mathit{false}$ & $\mathit{true}$ & $\mathit{false}$ & $\mathit{true}$ & $\mathit{false}$ & $\mathit{true}$ \\
  $\mathit{true}$ & $\bot$ & $\bot$ & $\bot$ & $\mathit{true}$ & $\bot$ & $\bot$ \\
  $\mathit{false}$ & $\mathit{true}$ & $\mathit{false}$ & $\mathit{false}$ & $\mathit{true}$ & $\mathit{true}$ & $\mathit{true}$ \\
  $\mathit{false}$ & $\mathit{false}$ & $\mathit{true}$ & $\mathit{false}$ & $\mathit{false}$ & $\mathit{true}$ & $\mathit{false}$ \\
  $\mathit{false}$ & $\bot$ & $\bot$ & $\mathit{false}$ & $\bot$ & $\mathit{true}$ & $\bot$ \\
  $\bot$ & $\mathit{true}$ & $\mathit{false}$ & $\bot$ & $\mathit{true}$ & $\mathit{true}$ & $\bot$ \\
  $\bot$ & $\mathit{false}$ & $\mathit{true}$ & $\mathit{false}$ & $\bot$ & $\bot$ & $\bot$ \\
  $\bot$ & $\bot$ & $\bot$ & $\bot$ & $\bot$ & $\bot$ & $\bot$ \\ \hline
\end{tabular}
\section{Real}
\subsection{Addition}
$+(y:\mathit{Real}):\mathit{Real}$ describes the sum of $\mathit{self}$ and $y$.\\
{\bf Notation:} $\mathit{self}$\verb-+-$y$
\subsection{Subtraction}
$-(y:\mathit{Real}):\mathit{Real}$ describes the difference between
$\mathit{self}$
  and $y$.\\
  {\bf Notation:} $\mathit{self}$\verb+-+$y$
\subsection{Multiplication}
$*(y:\mathit{Real}):\mathit{Real}$ describes the product of $\mathit{self}$ and
$y$.\\
{\bf Notation:} $\mathit{self}$\verb+*+$y$
\subsection{Division}
$/(y:\mathit{Real}):\mathit{Real}$ describes the quotient of $\mathit{self}$
and $y$.\\
{\bf Notation:} $\mathit{self}$\verb+/+$y$
\subsection{Negation}
$-():\mathit{Real}\stackrel{def}{=} 0-\mathit{self}$ describes the negation
of $\mathit{self}$.\\
{\bf Notation:} \verb+-+$\mathit{self}$
\subsection{Less}
$<(y:\mathit{Real}):\mathit{Boolean}$ evaluates to $\mathit{true}$,
if the value of
  $\mathit{self}$ is less than the value of $y$.  It evaluates to $\mathit{false}$, if
  the value of $\mathit{self}$ is equal to or greater than the value of $y$.
  In any other case, it is undefined.\\
  {\bf Notation:} $\mathit{self}$\verb+<+$y$
\subsection{Greater}
$>(y:\mathit{Real}):\mathit{Boolean}$ evaluates to $\mathit{true}$, if the value
of
  $\mathit{self}$ is greater than the value of $y$.  It evaluates to $\mathit{false}$,
  if the value of $\mathit{self}$ is less than or equal to the value of $y$.
  In any other case, it is undefined.\\
  {\bf Notation:} $\mathit{self}$\verb+>+$y$
\subsection{Less or equal}
$<=(y:\mathit{Real}):\mathit{Boolean}$ evaluates to $\mathit{true}$, if the
value of
  $\mathit{self}$ is less than or equal to the value of $y$.  It evaluates to
  $\mathit{false}$, if the value of $\mathit{self}$ is greater than the value of $y$.  In
  any other case, it is undefined.\\
  {\bf Notation:} $\mathit{self}$\verb+<=+$y$
\subsection{Greater or equal}
$>=(y:\mathit{Real}):\mathit{Boolean}$ evaluates to $\mathit{true}$, if the
value of
  $\mathit{self}$ is equal to or greater than the value of $y$.  It evaluates
  to $\mathit{false}$, if the value of $\mathit{self}$ is less than the value of $y$.
  In any other case, it is undefined.\\
  {\bf Notation:} $\mathit{self}$\verb+>=+$y$
\subsection{Absolute Values}
$\mathit{abs}():\mathit{Real}\stackrel{def}{=}\mathit{if} \mathit{self} < 0~ \mathit{then} -\mathit{self}\
\mathit{else}\ \mathit{self}\ \mathit{endif}$ describes the absolute value of $\mathit{self}$.\\
{\bf Notation:} $\mathit{self}$\verb+.abs()+
\subsection{Floor}
$\mathit{floor}():\mathit{Integer}$ describes the largest integer
which is not
  greater than $\mathit{self}$.\\
  {\bf Notation:} $\mathit{self}$\verb+.floor()+\\
  {\bf Note:} $\mathit{floor}$ binds stronger than $-$. That means
  $(-3.3).floor() = -4$ and $-3.3.floor() = -(3.3.floor()) = -3$.
\subsection{Round}
$\mathit{round}():\mathit{integer}\stackrel{def}{=}(\mathit{self}+0.5).\mathit{floor}()$ rounds
  $\mathit{self}$ to the nearest integer.\\
  {\bf Notation:} $\mathit{self}$\verb+.round()+
\subsection{Maximum}
$\mathit{max}(y:\mathit{Real}):\mathit{Real}\stackrel{def}{=}\mathit{if}\ \mathit{self} < y\ \mathit{then} y\
\mathit{else}\ \mathit{self}\ \mathit{endif}$ results in the greater value of $\mathit{self}$ and $y$.\\
  {\bf Notation:} $\mathit{self}$\verb+.max(+$y$\verb+)+
\subsection{Minimum}
$\mathit{min}(y:\mathit{Real}):\mathit{Real}\stackrel{def}{=}\mathit{if}\ \mathit{self}\ > y\ \mathit{then}\ y\
\mathit{else}\
  \mathit{self}\ \mathit{endif}$ results in the smaller value of $\mathit{self}$ and $y$.\\
  {\bf Notation:} $\mathit{self}$\verb+.min(+$y$\verb+)+
\section{Integer}
\subsection{Addition} $+(y:\mathit{Integer}):\mathit{Integer}$ describes the sum of
$\mathit{self}$ and
$y$.\\
  {\bf Notation:} $\mathit{self}$\verb-+-$y$
\subsection{Subtraction}
$-(y:\mathit{Integer}):\mathit{Integer}$ describes the difference
between $\mathit{self}$ and $y$.\\
  {\bf Notation:} $\mathit{self}$\verb+-+$y$
\subsection{Multiplication}
$*(y:\mathit{Integer}):\mathit{Integer}$ describes the product of $\mathit{self}$
and
  $y$.\\
  {\bf Notation:} $\mathit{self}$\verb+*+$y$
\subsection{Division}
$/(y:\mathit{Integer}):\mathit{Real}$ describes the quotient of $\mathit{self}$
and
  $y$.\\
  {\bf Notation:} $\mathit{self}$\verb+/+$y$
\subsection{Negation}
$-():\mathit{Integer}\stackrel{def}{=} 0-\mathit{self}$ describes the negation of
  $\mathit{self}$.\\
  {\bf Notation:} \verb+-+$\mathit{self}$
\subsection{Less}
$<(y:\mathit{Integer}):\mathit{Boolean}$ evaluates to $\mathit{true}$, if the
value of
  $\mathit{self}$ is less than the value of $y$.  It evaluates to $\mathit{false}$, if
  the value of $\mathit{self}$ is equal to or greater than the value of $y$.
  In any other case, it is undefined.\\
  {\bf Notation:} $\mathit{self}$\verb+<+$y$
\subsection{Greater}
$>(y:\mathit{Integer}):\mathit{Boolean}$ evaluates to $\mathit{true}$, if the
value of
  $\mathit{self}$ is greater than the value of $y$.  It evaluates to $\mathit{false}$,
  if the value of $\mathit{self}$ is less than or equal to the value of $y$.
  In any other case, it is undefined.\\
  {\bf Notation:} $\mathit{self}$\verb+>+$y$
\subsection{Less or equal}
$<=(y:\mathit{Integer}):\mathit{Boolean}$ evaluates to $\mathit{true}$,
if the
value of
  $\mathit{self}$ is less than or equal to the value of $y$.  It evaluates to
  $\mathit{false}$, if the value of $\mathit{self}$ is greater than the value of $y$.  In
  any other case, it is undefined.\\
  {\bf Notation:} $\mathit{self}$\verb+<=+$y$
\subsection{Greater or equal}
$>=(y:\mathit{Integer}):\mathit{Boolean}$ evaluates to $\mathit{true}$, if the
value of
  $\mathit{self}$ is equal to or greater than the value of $y$.  It evaluates
  to $\mathit{false}$, if the value of $\mathit{self}$ is less than the value of $y$.
  In any other case, it is undefined.\\
  {\bf Notation:} $\mathit{self}$\verb+>=+$y$
\subsection{Absolute Values}
$\mathit{abs}():\mathit{Integer}\stackrel{def}{=}\mathit{if}\ \mathit{self}\ < 0\ \mathit{then}\ -\mathit{self}\
\mathit{else}\
  \mathit{self}\ \mathit{endif}$ describes the absolute value of $\mathit{self}$.\\
  {\bf Notation:} $\mathit{self}$\verb+.abs()+
\subsection{Euclidean division}
$\mathit{div}(y:\mathit{Integer}):\mathit{Integer}$ describes Euclidean division
of
  $\mathit{self}$ by $y$, that is, it results in the unique integer $z$ such
  that there exists an $0\leq r < y$ with $z* y + r=\mathit{self}$.\\
  {\bf Notation:} $\mathit{self}$\verb+ div +$y$
\subsection{Modulo}
$\mathit{mod}(y:\mathit{Integer}):\mathit{Integer}$ describes
Euclidean reminder of
  $\mathit{self}$ divided by $y$, that is, it results in the unique integer
  $0\leq r < y$ such that there exists an integer $z$ with $z* y +
  r=\mathit{self}$.\\
  {\bf Notation:} $\mathit{self}$\verb+.mod(+$y$\verb+)+
\subsection{Maximum} $\mathit{max}(y:\mathit{Integer}):\mathit{Integer}\stackrel{def}{=}\mathit{if}\
\mathit{self}\ < y\ \mathit{then}\
y\ \mathit{else}\
  \mathit{self}\ \mathit{endif}$ evaluates to the greater value of $\mathit{self}$ and $y$.\\
  {\bf Notation:} $\mathit{self}$\verb+.max(+$y$\verb+)+
\subsection{Minimum} $\mathit{min}(y:\mathit{Integer}):\mathit{Integer}\stackrel{def}{=}\mathit{if}\
\mathit{self} > y\ \mathit{then}\
y\
  \mathit{else}\ \mathit{self}\ \mathit{endif}$ evaluates to the smaller value of $\mathit{self}$
  and $y$.\\
  {\bf Notation:} $\mathit{self}$\verb+.min(+$y$\verb+)+
\section{Collection}
$\langle\mathit{col}\rangle \Coloneqq \mathit{Set}\ |\ \mathit{Bag}\ |\ \mathit{Sequence}$
\subsection{Size}
$\mathit{size}():\mathit{Integer}\stackrel{def}{=}\mathit{self}\rightarrow\mathit{iterate}(e;a:\mathit{Integer}=0\mid
a+1)$\\
  {\bf Notation:} $\mathit{self}$\verb+->size()+
\subsection{Count}
$\mathit{count}(,y:T):\mathit{Integer}\stackrel{def}{=}
  \mathit{self}\rightarrow\mathit{iterate}(i;a:\mathit{Iterate}=0\mid\mathit{if}\ y=i\ \mathit{then}\ a+1\ \mathit{else}\
  a\ \mathit{endif})$ counts how often $y$ occurs in the collection.\\
  {\bf Notation:} $\mathit{self}$\verb+->count(+$y$\verb+)+
\subsection{Includes}
$\mathit{includes}(,y:T):\mathit{Boolean}\stackrel{def}{=}\mathit{self}\rightarrow\mathit{count}(y)>0$
  returns true if and only if $y$ occurs in the collection.\\
  {\bf Notation:} $\mathit{self}$\verb+->includes(+$y$\verb+)+
\subsection{Excludes}
$\mathit{excludes}(,y:T):\mathit{Boolean}\stackrel{def}{=}\mathit{self}\rightarrow\mathit{count}(y)=0$
  returns true if and only if $y$ does not occur in the collection.\\
  {\bf Notation:} $\mathit{self}$\verb+->excludes(+$y$\verb+)+
\subsection{Includes all}
$\mathit{includesAll}(,y:\langle\mathit{col}\rangle(T)):\mathit{Boolean}\stackrel{def}{=}
  y\rightarrow\mathit{forAll}(e\mid \mathit{self}\rightarrow\mathit{includes}(e))$.\\
  {\bf Notation:} $\mathit{self}$\verb+->includesAll(+$y$\verb+)+
\subsection{Excludes all}
$\mathit{excludesAll}(,y:\langle\mathit{col}\rangle(T)):\mathit{Boolean}\stackrel{def}{=}
  y\rightarrow\mathit{forAll}(e\mid \mathit{self}\rightarrow\mathit{excludes}(e))$.\\
  {\bf Notation:} $\mathit{self}$\verb+->excludesAll(+$y$\verb+)+
\subsection{Is empty}
$\mathit{isEmpty}():\mathit{Boolean}\stackrel{def}{=}\mathit{self}\rightarrow\mathit{size}()=0$.\\
  {\bf Notation:} $\mathit{self}$\verb+->isEmpty()+
\subsection{Not empty}
$\mathit{notEmpty}():\mathit{Boolean}\stackrel{def}{=}\mathit{self}\rightarrow\mathit{size}()<>0$.\\
  {\bf Notation:} $\mathit{self}$\verb+->notEmpty()+
\subsection{Sum}
$\mathit{sum}():T\stackrel{def}{=}\mathit{self}\rightarrow\mathit{iterate}(e;a:\mathit{Iterate}=0\mid a +
e)$.\\
 {\bf Notation:} $\mathit{self}$\verb+->sum()+
\section{Set}
\subsection{Set-Equality}
$=(y:\mathit{Set}(T)):\mathit{Boolean}$ describes set-equality.\\
  {\bf Notation:} $\mathit{self}$\verb+=+$y$
\subsection{Including elements}
$\mathit{including}(y:T):\mathit{Set}(T)$ describes the set obtained from
  $\mathit{self}$ by including $y$.\\
  {\bf Notation:} $\mathit{self}$\verb+->including(+$y$\verb+)+
\subsection{Excluding elements}
$\mathit{excluding}(y:T):\mathit{Set}(T)$ describes the set obtained from
  $\mathit{self}$ by excluding $y$.\\
  {\bf Notation:} $\mathit{self}$\verb+->excluding(+$y$\verb+)+
\subsection{Union}
$\mathit{union}(y:\mathit{Set}(T)):\mathit{Set}(T)$ describes the union of the
set
  $\mathit{self}$ and the set $y$.\\
  {\bf Notation:} $\mathit{self}$\verb+->union(+$y$\verb+)+
\subsection{Union with Bag}
$\mathit{union}(y:\mathit{Bag}(T)):\mathit{Bag}(T)$ describes the union of the
set
  obtained from $\mathit{self}$ by assuming that each element of $\mathit{self}$
  occurs exactly once and the bag $y$.\\
  {\bf Notation:} $\mathit{self}$\verb+->union(+$y$\verb+)+
\subsection{Intersection}
$\mathit{intersection}(y:\mathit{Set}(T)):\mathit{Set}(T)$ describes the
intersection
  of the set $\mathit{self}$ the set $y$.\\
  {\bf Notation:} $\mathit{self}$\verb+->intersection(+$y$\verb+)+
\subsection{Intersection with Bag}
$\mathit{intersection}(y:\mathit{Bag}(T)):\mathit{Set}(T)$ describes the
intersection
  of the set $\mathit{self}$ and the set obtained from $y$ by including every
  element contained in $y$.\\
  {\bf Notation:} $\mathit{self}$\verb+->intersection(+$y$\verb+)+
\subsection{Difference of sets}
$-(y:\mathit{Set}(T)):\mathit{Set}(T)$ describes the difference of the set
  $\mathit{self}$ the set $y$.\\
  {\bf Notation:} $\mathit{self}$\verb+-+$y$
\subsection{Flatten}
$\mathit{flatten}():\mathit{Set}(T')$.  If $\mathit{self}$ is a set of
collections,
  then this operation returns the set-union of all its elements.\\
  {\bf Notation:} $\mathit{self}$\verb+->flatten()+
%\subsection{As Set}
%!!!!! (gibt es in USE nicht. Soll es hinzugef�gt werden?) $\mathit{asSet}():
%  \mathit{Set}(T)\stackrel{def}{=}\mathit{self}\rightarrow\mathit{iterate}(e;a:\mathit{Set}(T)=\mathit{oclEmpty}(\mathit{Set}(T))\mid
%  a\rightarrow\mathit{including}(e))$ is the identity function.\\
%  {\bf Notation:} $\mathit{self}$\verb+->asSet()+
\subsection{As Bag}
$\mathit{asBag}():
  \mathit{Bag}(T)\stackrel{def}{=}\mathit{self}\rightarrow\mathit{iterate}(e;a:\mathit{Bag}(T)=\mathit{oclEmpty}(\mathit{Bag}(T))\mid
  a\rightarrow\mathit{including}(e))$ returns a bag which includes each
  element of $\mathit{self}$ exactly once.\\
  {\bf Notation:} $\mathit{self}$\verb+->asBag()+
\subsection{As Sequence}
$\mathit{asSequence}(): \mathit{Sequence}(T)$ returns a sequence
containing
  all elements of $\mathit{self}$ exactly once.  The order of the elements is
  arbitrary.  It is equivalent to the expression
  \begin{equation*}
    {\mathit{self}\rightarrow\mathit{iterate}(e;a:\mathit{Sequence}(T)=\mathit{oclEmpty}(\mathit{Sequence}(T)\mid
      a\rightarrow\mathit{append}(e))}\quad.
  \end{equation*}\\
  {\bf Notation:} $\mathit{self}$\verb+->asSequence()+
\section{Bag}
\subsection{Equality}
$=(y:\mathit{Bag}(T)):\mathit{Boolean}$ describes equality of
multi-sets.\\
  {\bf Notation:} $\mathit{self}$\verb+=+$y$
\subsection{Including elements}
$\mathit{including}(y:T):\mathit{Bag}(T)$ describes the bag
obtained from
  $\mathit{self}$ by including $y$.\\
  {\bf Notation:} $\mathit{self}$\verb+->including(+$y$\verb+)+
\subsection{Excluding elements}
$\mathit{excluding}(y:T):\mathit{Bag}(T)$ describes the bag obtained from
  $\mathit{self}$ by excluding all occurrences of $y$.\\
  {\bf Notation:} $\mathit{self}$\verb+->excluding(+$y$\verb+)+
\subsection{Union}
$\mathit{union}(y:\mathit{Bag}(T)):\mathit{Bag}(T)$ describes the union of the
bag
  $\mathit{self}$ and the bag $y$.\\
  {\bf Notation:} $\mathit{self}$\verb+->union(+$y$\verb+)+
\subsection{Union with Set}
$\mathit{union}(y:\mathit{Set}(T)):\mathit{Bag}(T)$ describes the union of the
bag
  $\mathit{self}$ and the set obtained from $y$ by including each element of
  $y$ exactly once.\\
  {\bf Notation:} $\mathit{self}$\verb+->union(+$y$\verb+)+
\subsection{Intersection}
$\mathit{intersection}(y:\mathit{Bag}(T)):\mathit{Bag}(T)$ describes the
intersection
  of the bag $\mathit{self}$ and the bag $y$.\\
  {\bf Notation:} $\mathit{self}$\verb+->intersection(+$y$\verb+)+
\subsection{Intersection with Set}
$\mathit{intersection}(y:\mathit{Set}(T)):\mathit{Set}(T)$ describes the
intersection
  of the bag $\mathit{self}$ and the set $y$.\\
  {\bf Notation:} $\mathit{self}$\verb+->intersection(+$y$\verb+)+
%\subsection{Difference of bags}
%!!!!! (existiert nicht in USE; hinzuf�gen?) $-(y:\mathit{Bag}(T)):\mathit{Bag}(T)$ describes the bag obtained from
%  $\mathit{self}$ by removing all occurrences of the elements of $y$.\\
%  {\bf Notation:} $\mathit{self}$\verb+->union(+$y$\verb+)+
\subsection{Flatten}
$\mathit{flatten}():\mathit{Bag}(T')$.  If $\mathit{self}$ is a bag of
collections,
  then this operation returns the bag union of all its elements.\\
  {\bf Notation:} $\mathit{self}$\verb+->flatten(+$y$\verb+)+
\subsection{As Set}
$\mathit{asSet}():
  \mathit{Set}(T)\stackrel{def}{=}\mathit{self}\rightarrow\mathit{iterate}(e;a:\mathit{Set}(T)=\mathit{oclEmpty}(\mathit{Set}(T))\mid
  a\rightarrow\mathit{including}(e))$ returns a set which contains each
  element of $\mathit{self}$.\\
  {\bf Notation:} $\mathit{self}$\verb+->asSet()+
%\subsection{As Bag}
%!!!!! (existiert nicht; hinzuf�gen?) $\mathit{asBag}():\mathit{Bag}(T)$ is the identity function.\\
%  {\bf Notation:} $\mathit{self}$\verb+->union(+$y$\verb+)+
\subsection{As Sequence}
$\mathit{asSequence}():\mathit{Sequence}(T)$ returns a sequence containing
  all elements of $\mathit{self}$ as often as they occur in the multi-set.
  The order of the elements is arbitrary. It is equivalent to:
  \begin{equation*}
    \mathit{self}\rightarrow\mathit{iterate}(e;a:\mathit{Sequence}(T)=\mathit{oclEmpty}(\mathit{Sequence}(T))\mid
    a\rightarrow\mathit{append}(e))\quad.
  \end{equation*}
  {\bf Notation:} $\mathit{self}$\verb+->union(+$y$\verb+)+
\section{Sequence}
\subsection{Get element}
$\mathit{at}(y:\mathit{Integer}):T$ results in the element at the $y$th
  position of the sequence.\\
  {\bf Notation:} $\mathit{self}$\verb+->at(+$y$\verb+)+
\subsection{Equality}
$=(y:\mathit{Sequence}(T)):\mathit{Boolean}\stackrel{def}{=}\mathit{let}\
  s={\mathit{self}\rightarrow\mathit{size}()}\ \mathit{in}\
  s={y\rightarrow\mathit{size}()}\ \linebreak[1]\mathit{and}\ \linebreak[2]
  \mathit{Sequence}\{1..s\}\rightarrow
  \mathit{forAll}(i:\mathit{Integer}\mid \mathit{self}\rightarrow\mathit{at}(i)=y\rightarrow\mathit{at}(i))$
  describes equality of sequences.\\
  {\bf Notation:} $\mathit{self}$\verb+=+$y$
\subsection{Union}
$\mathit{union}(y:\mathit{Sequence}(T)):\mathit{Sequence}(T)$ describes the
  concatenation of $\mathit{self}$ and $y$.\\
  {\bf Notation:} $\mathit{self}$\verb+->union(+$y$\verb+)+
\subsection{Flatten}
$\mathit{flatten}(\mathit{self}:\mathit{Sequence}(T)):\mathit{Sequence}(T')$.  If $\mathit{self}$ is a sequence of
  collections, then this operation returns the sequence concatenation
  of all its elements.\\
  {\bf Notation:} $\mathit{self}$\verb+->flatten()+
\subsection{Append elements}
$\mathit{append}(y:T):\mathit{Sequence}(T)\stackrel{def}{=}
\mathit{self}\rightarrow\mathit{union}(\mathit{Sequence}\{y\})$ results in the sequence
  which consists of all elements of $y$ with $y$ appended.\\
  {\bf Notation:} $\mathit{self}$\verb+->append(+$y$\verb+)+
\subsection{Prepend elements}
$\mathit{prepend}(y:T):\mathit{Sequence}(T)\stackrel{def}{=}
  \mathit{Sequence}\{y\}\rightarrow\mathit{union}(\mathit{self})$ results in the sequence
  which consists of all elements of $\mathit{self}$ with $y$ prepended.\\
  {\bf Notation:} $\mathit{self}$\verb+->prepend(+$y$\verb+)+
\subsection{Excluding elements}
$\mathit{excluding}(y:T):\mathit{Sequence}(T)\stackrel{def}{=} \mathit{self}\rightarrow
  \mathit{iterate}(i;a:\mathit{Sequence}(T)=\mathit{oclEmpty}(\mathit{Sequence}(T))\mid \mathit{if}\
  y=i\ \mathit{then}\ a\ \linebreak[2]
  \mathit{else}\ a\rightarrow\mathit{append}(i)\ \mathit{endif})$ results in
  the largest sub-sequence of $\mathit{self}$, in which $y$ does not occur.\\
  {\bf Notation:} $\mathit{self}$\verb+->excluding(+$y$\verb+)+
\subsection{Subsequence}
$\mathit{subSequence}(y:\mathit{Integer},z:\mathit{Integer}):\mathit{Sequence}(T)$
  results in the subsequence of $\mathit{self}$ starting at index $y$ and
  ending at index $z$.  It is equivalent to:
  \begin{multline*}
    \mathit{Sequence}\{1..z\}\rightarrow\mathit{Iterate}(i;
    a:\mathit{Sequence}(T)=\mathit{oclEmpty}(\mathit{Sequence}(T))\mid{}\\
    \mathit{if}\ y\leq i\ \mathit{and}\ i\leq z\ \mathit{then}\
    a\rightarrow\mathit{append}(\mathit{self}\rightarrow\mathit{at}(i))\mathit{else}\ a\ \mathit{endif})
  \end{multline*}\\
  {\bf Notation:} $\mathit{self}$\verb+->subSequence(+$y$\verb+)+
%\subsection{Insert element}
%!!!!! (existiert nicht, hinzuf�gen?)
%$\mathit{insertAt}(y:T,z:\mathit{Integer}):\mathit{Sequence}(T)\stackrel{def}{=}
%  (\mathit{self}\rightarrow\mathit{subSequence}(1,z-1)\rightarrow\mathit{append}(y))\linebreak[0]
%  \rightarrow\mathit{union}(\mathit{self}\rightarrow
%  \mathit{Subsequence}(z,\mathit{self}\rightarrow\mathit{size}()))$ results in a sequence
%  where $y$ is inserted at position $z$.\\
%  {\bf Notation:} $\mathit{self}$\verb+->union(+$y$\verb+)+
\subsection{Get first element}
$\mathit{first}():T\stackrel{def}{=}\mathit{self}\rightarrow\mathit{at}(1)$.\\
  {\bf Notation:} $\mathit{self}$\verb+->first(+$y$\verb+)+
\subsection{Get last element}
$\mathit{last}():T\stackrel{def}{=}\mathit{self}\rightarrow\mathit{at}(\mathit{self}\rightarrow\mathit{size}())$.\\
  {\bf Notation:} $\mathit{self}$\verb+->last(+$y$\verb+)+
%\subsection{As Sequence}
%!!!!! (existiert nicht, hinzuf�gen?)$\mathit{asSequence}(): \mathit{Sequence}(T)$ is the identity function.\\
%  {\bf Notation:} $\mathit{self}$\verb+->union(+$y$\verb+)+
\subsection{As Set}
$\mathit{asSet}():
  \mathit{asSet}(T)\stackrel{def}{=}\mathit{self}\rightarrow\mathit{iterate}(e;a:\mathit{Set}(T)=\mathit{oclEmpty}(\mathit{Set}(T))\mid
  a\rightarrow\mathit{including}(e))$ returns a set which contains each
  element of $\mathit{self}$.\\
  {\bf Notation:} $\mathit{self}$\verb+->asSet(+$y$\verb+)+
\subsection{As Bag}
$\mathit{asBag}():
  \mathit{Bag}(T)\stackrel{def}{=}\mathit{self}\rightarrow\mathit{iterate}(e;a:\mathit{Bag}(T)=\mathit{oclEmpty}(\mathit{Bag}(T))\mid
  a\rightarrow\mathit{including}(e))$ returns a bag containing all elements
  of the sequence $\mathit{self}$.\\
  {\bf Notation:} $\mathit{self}$\verb+->asBag(+$y$\verb+)+
